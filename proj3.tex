\documentclass[11pt,a4paper]{article}
\usepackage[utf8]{inputenc}
\usepackage[czech]{babel}
\usepackage[T1]{fontenc}
\usepackage[left=2cm,top=3cm,text={17cm, 24cm}]{geometry}
\usepackage{times}
\author{Vladan Kudláč}

\usepackage{amsmath}
\usepackage{amsfonts}
\usepackage{amssymb}
\usepackage{graphics}
\usepackage{picture}

\usepackage[czech,linesnumbered,ruled,longend,noline]{algorithm2e}
\usepackage{multirow}
\usepackage{pdflscape}

\begin{document}

\begin{titlepage}
\begin{center}
	\textsc{
		{\Huge Vysoké učení technické v~Brně} \\
		\medskip
		{\huge Fakulta informačních technologií} \\
	}
	\vspace{\stretch{0.1}}
	\includegraphics{logoFIT.eps}
	\vspace{\stretch{0.282}}
	
	{\LARGE Počítačové komunikace a sítě\,--\,1. projekt} \\
	\medskip
	{\Huge Klient-server pro jednoduchý přenos souborů}
	\vspace{\stretch{0.618}}
	
	{\Large 5. března 2018 \hfill Vladan Kudláč}
\end{center}
\end{titlepage}

\tableofcontents
\pagebreak

\section{Úvod}
Prohlašuji, že jsem tuto semestrální práci vypracoval samostatně, uvedl jsem všechny literární prameny a publikace, ze kterých jsem čerpal.
Dokumentace, uživatelská příručka a vestavěná nápověda je psána v češtině. Programová dokumentace je stejně jako samotný kód psána v angličtině a není součástí odevzdávané dokumentace.

\section{Aplikační protokol}
Poté, co klient naváže se serverem spojení, zašle klient serveru požadavek v jednom ze dvou následujících tvarů:
\begin{itemize}
\item \texttt{\textbf{SEND <filename>}} -- soubor filename bude nahrán na server (klient->server)
\item \texttt{\textbf{RECV <filename>}} -- soubor filename bude stažen ze serveru (server->klient)
\end{itemize}
Zpráva obsahuje \textbf{klíčové slovo} \texttt{SEND} nebo \texttt{RECV}, \textbf{mezeru} (ASCII kód 32) a následuje \textbf{posloupnost libovolných znaků} určujících název souboru. Server klientu odpoví jedním ze stavových kódů:
\begin{itemize}
\item \texttt{\textbf{200 OK}} -- vše v pořádku, přenos může začít
\item \texttt{\textbf{403 FILE\_ERROR}} -- klient žádá o práci se souborem, který nelze otevřít
\item \texttt{\textbf{400 BAD\_REQUEST}} -- klient nesplnil předepsaný tvar požadavku
\end{itemize}
Odpovědi jsou inspirované stavovými kódy aplikačního protokolu HTTP. Jakákoliv jiná odpověď než \texttt{200 OK} je považována za chybu a k přenosu nedojde. Chybu je vhodné uživateli vypsat.

Pokud server vrátí kód 200, může být zahájen přenos. Od této chvíle je dohodnuto, která strana bude příjemcem a která odesílatelem, komunikace se stává jednosměrnou. Odesílatel odesílá data v binární podobě, příjemce data pouze přijímá. Jakmile odesílatel odešle všechna data, ukončí spojení.

\section{Programové řešení}
Výsledný program je implementovaný v jazyce C++ a není zpětně kompatibilní s jazykem C, předpokládají se pouze adresy IPv4.

\section{Uživatelská příručka}
\subsection{Překlad}
\subsection{Spuštění}
\subsection{Chybové kódy}

\section{Závěr}

\end{document}
